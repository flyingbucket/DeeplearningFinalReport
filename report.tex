\documentclass[aspectratio=169]{beamer}
\usetheme{metropolis}
\metroset{subsectionpage=progressbar}

\usefonttheme{professionalfonts} % 使用系统/自定字体


% === 字体设置 ===
\usepackage[UTF8,scheme=plain,fontset=none]{ctex}
\setCJKmainfont{Source Han Serif CN}[BoldFont={Source Han Serif CN Bold}]
\setCJKsansfont{Source Han Sans CN}[BoldFont={Source Han Sans CN Bold}]
% \setCJKmonofont{Sarasa Mono CN}

% \input{commands.tex}

% beamer 已加载 hyperref;加 unicode 以支持中文书签
\hypersetup{unicode}

% define paragraph
\providecommand{\paragraph}[1]{\smallskip\textbf{#1}\par}

% 常用包
\usepackage{longtable,booktabs}
\usepackage{amsmath,amssymb}
\usepackage{graphicx}
\usepackage{tikz}
\usetikzlibrary{positioning,arrows.meta}
% \graphicspath{{.}{./figs/}{./images/}{./images_in_paper/}}
\usepackage{caption}
\usepackage{subcaption}
\usepackage{float}
\usepackage{svg}
\usepackage{booktabs}
\usepackage{array}
\usepackage{threeparttable}

% 算法环境(与 beamer 兼容)
\usepackage{algorithm}
\usepackage[noend]{algpseudocode}  % 提供 algorithmic 环境、\State 等
% 可选:微调 algorithmic 缩进
\algrenewcommand\algorithmicindent{0.8em}

% 数学粗体与梯度符号
\usepackage{bm} % \boldsymbol
\newcommand{\bx}{\mathbf{x}}
\newcommand{\bz}{\mathbf{z}}
\newcommand{\bI}{\mathbf{I}}
\newcommand{\bzero}{\mathbf{0}}
\newcommand{\bepsilon}{\boldsymbol{\epsilon}}
\newcommand{\grad}{\nabla}
% 超链接(beamer 已加载 hyperref,这里只补选项)
% \hypersetup{unicode=true}

% 编号风格
\setbeamertemplate{caption}[numbered]
\setbeamertemplate{caption label separator}{.}

\title{基于transformer的diffusion超分辨模型}
\author{李孟霖}
\date{\today}

%---Document Begins---
\begin{document}
\begin{frame}[plain]
  \titlepage
\end{frame}
\section{Introduction}
\subsection{研究背景}
\begin{frame}{主流超分辨模型}
  EDSR(CNN)
 
  SwinIR(transformer)
 
  StableSR(Diffusion) 
 
  Diffusion模型在生成任务中往往具有比较好的高频细节生成能力,
  本研究中考虑使用Diffusion方法构造超分辨模型
\end{frame}

\begin{frame}{Diffusion backbone}
  UNet 
 
  DiT 
 
  Transformer模型相比于UNet具有更好的全局性、更强的灵活性和更低的计算开销,
  本研究考虑使用Transformer作为去噪模块
\end{frame}

\begin{frame}{去噪方式}
  epsilon-prediction 
 
  x-prediction 
 
  v-prediction 
\end{frame}
\subsection{相关工作}
\begin{frame}{JiT}
  何恺明团队CFG条件生成模型(JiT),Back to Basics: Let Denoising Generative Models Denoise

  低维流形假设

  以vit为去噪模块backbone,模型做x-prediction,x-prediction的预测值变换为v-prediction构造损失函数
\end{frame}

\section{Methods}
\begin{frame}{模型概览}
  主体与JiT相同,删去CFGtoken拼接,尝试按下述两种方式将低清图信息引入 
\end{frame}

\begin{frame}{低清图插入方式}
  \begin{itemize}
    \item 将低清图编码为token,和带噪声图像拼接成一个序列
    \item 将低清图和带噪声图像按通道拼接
  \end{itemize}
\end{frame}

\section{Experiments}
\subsection{按token拼接}
\begin{frame}{性能指标}
  psnr ssim 
\end{frame}
\begin{frame}
  效果图 
\end{frame}
\subsection{按通道拼接}
\begin{frame}{性能指标}
  psnr ssim 
\end{frame}
\begin{frame}
  效果图 
\end{frame}

\subsection{基线对比}


\begin{frame}[standout]
  谢谢大家!
\end{frame}
\end{document}
